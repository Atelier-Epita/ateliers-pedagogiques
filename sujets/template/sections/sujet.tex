\section{Sujet}
Développez ici l'entièreté de votre sujet.
Structurez de manière cohérente celui-ci en utilisant des subsections, subsubsections, ...

\subsection{Encadrements}

\warning{
Afin de notifier un point important :
\mono{\textbackslash warning\{\textit{texte}\}}
}

\hint{
Pour ajouter une information :
\mono{\textbackslash hint\{\textit{texte}\}}
}

\bonus{
Exercice bonus si la personne est suffisamment avancée :
\mono{\textbackslash bonus\{\textit{texte}\}}
}

\subsection{Saut de ligne}

Utilisez la commande \mono{\textbackslash br}

\subsection{Lien}
Pour intégrer un lien utilisez la commande
\mono{\textbackslash doclink\{\textit{lien}\}\{\textit{texte}\}}

\subsection{Morceau de code court}

Utilisez la commande \mono{\textbackslash mono\{\textit{texte}\}} pour afficher un texte stylisé (comme ici).

\subsection{Bloc de code}

Utilisez le package listings défini \doclink{https://texdoc.org/serve/listings.pdf/0}{ici}.
Par exemple,\\
\mono{\textbackslash begin\{lstlistings\}[language=...]} \hspace{6pt}
\textit{votre code} \hspace{6pt}
\mono{\textbackslash end\{lstlistings\}}\\
vous donne le rendu suivant :

\begin{lstlisting}[language=python, title={Exemple de bloc de code}]
import math

# important
def square(x):
	'''
	wow what a function!
	'''
	y = 0
	for _ in range(x):
		y += x
	return y
	
print("5^2 = ", str(square(5)))
\end{lstlisting}

Si vous souhaitez avoir un bloc sans rupture rajoutez les paramètres
\mono{float, floatplacement=H}.

\subsection{Illustration}

Pour ajouter une illustration (figure) au bon format
\mono{\textbackslash illustrate\{\textit{path}\}\{\textit{texte}\}}

\warning{Le \textit{path} est à prendre à partir du dossier \texttt{images/}.}