\section{Avant de commencer}

\subsection{Prérequis}

Il vous faudra un ordinateur et le logiciel \hyperref[download]{OpenSCAD} disponible sur MacOS, Windows et la plupart des distributions Linux.
Le logiciel demande peu de ressources matérielles et peut ainsi être utilisé sur beaucoup de configurations.

Aucune connaissance particulière n'est nécessaire, que ce soit en modélisation 3D ou en programmation.


\subsection{Présentation d'OpenSCAD}

OpenSCAD est un logiciel libre permettant de créer des objets solides de \textbf{C}onception \textbf{A}ssistée par \textbf{O}rdinateur (CAO) en 3D.
Il s'agit d'un modeleur basé uniquement sur des scripts qui utilise son propre langage de description.

Les pièces peuvent être prévisualisées, mais ne peuvent pas être modifiées de manière interactive avec la souris dans la vue 3D.
Un script OpenSCAD spécifie des primitives géométriques (telles que des sphères, des boîtes, des cylindres, etc...) et définit comment elles sont modifiées et combinées (par exemple par intersection, différence, combinaison, d'enveloppes et sommes de Minkowski) pour rendre un modèle 3D.
En tant que tel, le programme fait de la \textbf{G}éométrie \textbf{S}olide \textbf{C}onsructive (GSC).

Ainsi, les sauvegardes ne sont que des fichiers textes en clair, ce qui facilite grandement le partage, en plus du volume de stockage très faible.