\section{Annexes}

\subsection{Corrigé des exercices d'application}

\illustrate{exo_1.png}{Envie de buzzer...}

\illustrate{exo_2.png}{Que la lumière soit !}

\illustrate{exo_3.png}{Bientôt DJ ?}

\begin{tcolorbox}[colback=yellow-atelier!10, colframe=gray!70!black, coltitle=black, colbacktitle=gray!70!white, title={\textit{Envie de buzzer...}}, leftrule=2mm]
\begin{small}
\begin{verbatim}
const int button = 1, piezo = 3;

bool is_pressed = false;

void setup(){
  pinMode(button, INPUT);
  pinMode(piezo, OUTPUT);
}

void loop(){  
  if (!is_pressed && digitalRead(button) == HIGH){
    is_pressed = true;
    analogWrite(piezo, 20); 	// doit être >0 et <1024
  }
  else if (is_pressed && digitalRead(button) == LOW){
    is_pressed = false;
    analogWrite(piezo, 0);
  }
  
  delay(10);
}
\end{verbatim}
\end{small}
\end{tcolorbox}

\begin{tcolorbox}[colback=yellow-atelier!10, colframe=gray!70!black, coltitle=black, colbacktitle=gray!70!white, title={\textit{Que la lumière soit !}}, leftrule=2mm]
\begin{small}
\begin{verbatim}
const int button = 1, piezo = 3, led_g = 12, led_r = 13;

bool is_pressed = false;

void setup(){
  pinMode(button, INPUT);
  pinMode(piezo, OUTPUT);
  pinMode(led_g, OUTPUT);
  pinMode(led_r, OUTPUT);
}

void loop(){
  if (!is_pressed && digitalRead(button) == HIGH){
    is_pressed = true;
    analogWrite(piezo, 20); 	// doit être >0 et <1024
    digitalWrite(led_r, LOW);
    digitalWrite(led_g, HIGH);
  }
  else if (is_pressed && digitalRead(button) == LOW){
    is_pressed = false;
    analogWrite(piezo, 0);
    digitalWrite(led_g, LOW);
    digitalWrite(led_r, HIGH);
  }
  
  delay(10);
}
\end{verbatim}
\end{small}
\end{tcolorbox}

\begin{tcolorbox}[colback=yellow-atelier!10, colframe=gray!70!black, coltitle=black, colbacktitle=gray!70!white, title={\textit{Bientôt DJ ?}}, leftrule=2mm]
\begin{small}
\begin{verbatim}
const int button = 1, piezo = 3, led_g = 12, led_r = 13, potentiometer = A0;

bool is_pressed = false;
int frequency = 0;

void setup(){
  pinMode(button, INPUT);
  pinMode(piezo, OUTPUT);
  pinMode(led_g, OUTPUT);
  pinMode(led_r, OUTPUT);
}

void loop(){
  frequency = analogRead(potentiometer) *20 / 1023;
  
  if (!is_pressed && digitalRead(button) == HIGH){
    is_pressed = true;
    analogWrite(piezo, frequency);
    digitalWrite(led_r, LOW);
    digitalWrite(led_g, HIGH);
  }
  else if (is_pressed && digitalRead(button) == LOW){
    is_pressed = false;
    analogWrite(piezo, 0);
    digitalWrite(led_g, LOW);
    digitalWrite(led_r, HIGH);
  }
  
  delay(10);
}
\end{verbatim}
\end{small}
\end{tcolorbox}


\clearpage
\subsection{Correction circuit mini-projet arduiPONG}
\illustrate{pong.png}{Proposition de circuit répondant au sujet}


\clearpage
\subsection{Liens utiles}

\begin{itemize}
	\item \doclink{https://www.tinkercad.com}{Tinkercad}
	\item \doclink{https://docs.arduino.cc/}{Documentation Arduino}
	\item \doclink{https://www.electronics-tutorials.ws/}{Pour aller plus loin}
		  Ce site contient divers cours d'électronique très détaillés et facilement compréhensibles.
		  N'hésitez pas à y jeter un oeil pour comprendre plus en détail les concepts et composants abordés.
\end{itemize}