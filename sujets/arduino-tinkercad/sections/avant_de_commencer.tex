\section{Avant de commencer}
\subsection{Créer un nouveau circuit}
Vous aurez besoin d'un compte Autodesk pour suivre cet atelier.
Si vous n'en avez pas, suivez \doclink{https://www.tinkercad.com/join}{ce lien}.

Une fois que vous êtes connecté, allez sur \doclink{https://www.tinkercad.com/dashboard?collection=designs&type=circuits}{la section circuit} et créez en un nouveau que vous pourrez renommer à votre guise.

\subsection{Interface}
\illustrate{tinkercad.png}{Interface principale de Tinkercad en mode Circuit}

\begin{itemize}
	\item Sur la barre du haut vous pouvez
	\begin{itemize}
		\item copier/coller/supprimer des composants
		\item annuler/refaire des actions
		\item inclure/afficher des notes sur le circuit
		\item modifier la couleur et le type de câble
		\item pivoter/refléter un composant
	\end{itemize}
	\item En haut à droite, vous pouvez changer la vue (circuit ou schéma) et voir la liste des composants
	\item Sur la partie droite se situe la liste des composants disponibles ainsi que des petits circuits de départ (starters).
		  Vous pouvez alterner la vue en grille (par défaut) ou en détail pour avoir une description de chaque composant.
	\item La zone de travail est au centre de l'écran
	\item En haut à droite vous pouvez afficher la fenêtre de code en mode block ou textuel (nous n'utiliserons que le textuel dans cet atelier)
	\item Au même endroit, vous pouvez lancer la simulation.
		  Les différents composants seront utilisables de manière interactive mais vous ne pourrez pas modifier le circuit.
\end{itemize}