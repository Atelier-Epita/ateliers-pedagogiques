\section{Sujet}

\subsection{Fonctionnement et branchement des composants}

\subsubsection{Bouton poussoir}

Un bouton poussoir permet, lorsqu'il est pressé, de connecter deux sections de circuit entre elles.
En passant votre souris au dessus des pattes, vous pouvez voire un nom composé d'un chiffre (1 ou 2) et d'une lettre (a ou b).
Les pattes de même chiffre sont reliées entre elles en permanence.
Lorsque le bouton est pressé, les deux sections sont connectées au milieu.

\illustrate{pushbutton.png}{Les fils de même couleur sont connectés entre eux.}

\subsubsection{LED}

Une LED est une diode émettant de la lumière, ce qui signifie qu'elle ne peut être branchée que dans un sens uniquement.
La patte courte est la cathode (-) et la patte longue, qui est parfois courbée comme sur Tinkercad, est l'anode (+).

Il est toujours nécessaire de brancher une résistance en série avec une LED afin de contrôler la tension traversant celle-ci.
La valeur de la résistance se calcule simplement avec la \doclink{https://fr.wikipedia.org/wiki/Loi_d'Ohm}{loi d'Ohm}.
Si aucune résistance n'est branchée en série, il y a un très fort risque de griller la LED (allez essayer sur Tinkercad pour observer la réaction !).

\illustrate{led.png}{Branchement caractérisitque d'une LED}

\hint{Pour comprendre de façon beaucoup plus détaillée le fonctionnement d'une LED, vous pouvez lire \doclink{https://www.electronics-tutorials.ws/diode/diode_8.html}{cette page}.}

\subsubsection{Breadboard}

Une breadboard permet de réaliser le prototype d'un circuit électronique et de le tester. L'avantage de ce système est d'être totalement réutilisable, car il ne nécessite pas de soudure.

Les pins d'une même colonne sont tous liés ensemble.
Sur certains modèles, des bus communs sont présents sur les côtés.
Ils servent notamment à relier directement les composants à la masse (-) et au circuit d'alimentation (+).

Sur Tinkercad, vous pouvez voir les sections connectées ensemble en passant votre souris au dessus.

\illustrate{breadboard}{Deux breadboards. Celle de droite est équipée de bus communs.}