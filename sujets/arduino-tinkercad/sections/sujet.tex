\section{Sujet}

\subsection{Fonctionnement et branchement de quelques composants}

\subsubsection{Breadboard}

Une breadboard permet de réaliser le prototype d'un circuit électronique et de le tester. L'avantage de ce système est d'être totalement réutilisable, car il ne nécessite pas de soudure.

Les pins d'une même colonne sont tous liés ensemble.
Sur certains modèles, des bus communs sont présents sur les côtés.
Ils servent notamment à relier directement les composants à la masse (-) et au circuit d'alimentation (+).

Sur Tinkercad, vous pouvez voir les sections connectées ensemble en passant votre souris au dessus.

\illustrate{breadboard}{Deux breadboards ; celle de droite est équipée de bus communs}

\subsubsection{LED}

Une LED est une diode émettant de la lumière, ce qui signifie qu'elle ne peut être branchée que dans un sens uniquement.
La patte courte est la cathode (-) et la patte longue, qui est parfois courbée comme sur Tinkercad, est l'anode (+).

Il est toujours nécessaire de brancher une résistance en série avec une LED afin de contrôler la tension traversant celle-ci.
La valeur de la résistance se calcule simplement avec la \doclink{https://fr.wikipedia.org/wiki/Loi_d'Ohm}{loi d'Ohm}.
Si aucune résistance n'est branchée en série, il y a un très fort risque de griller la LED (allez essayer sur Tinkercad pour observer la réaction !).

\illustrate{led.png}{Branchement caractéristique d'une LED}

\hint{Pour comprendre de façon beaucoup plus détaillée le fonctionnement d'une LED, vous pouvez lire \doclink{https://www.electronics-tutorials.ws/diode/diode_8.html}{cette page}.}

\subsubsection{Bouton poussoir}

Un bouton poussoir permet, lorsqu'il est pressé, de connecter deux sections de circuit entre elles.
En passant votre souris au dessus des pattes, vous pouvez voir un nom composé d'un chiffre (1 ou 2) et d'une lettre (a ou b).
Les pattes de même chiffre sont reliées entre elles en permanence.
Lorsque le bouton est pressé, les deux sections sont connectées au milieu.

\illustrate{pushbutton.png}{Bouton poussoir : les fils de même couleur sont connectés entre eux}

\subsubsection{Piezo buzzer}

Un piezo génère du son en étant traversé par un courant électrique alternatif (AC).
Plus la fréquence de ce courant est élevée, plus le son produit sera aigu.
Plus l'intensité du courant est élevée, plus le son sera fort.

Ce composant est utile pour générer facilement de petits sons mais n'est pas adapté pour servir de haut-parleur.
Il est cependant petit et très peu cher.

\illustrate{piezo.png}{Buzzer utilisant l'effet piezo-électrique}


\clearpage
\subsection{Arduino}

Arduino est la marque d'une plateforme de prototypage open-source qui permet de créer des objets électroniques interactifs à partir de cartes électroniques sur lesquelles se trouve un microcontrôleur.
Le microcontrôleur peut être programmé pour analyser et produire des signaux électriques, de manière à effectuer des tâches très diverses comme la domotique, le pilotage d'un robot, de l'informatique embarquée, etc.
C'est une plateforme basée sur une interface entrée/sortie simple.

Vous allez utiliser dans cet atelier une carte \textbf{Arduino Uno R3} mais il en existe de nombreux autres modèles plus adaptés à certains cas d'utilisation, comme les cartes \textit{Nano} ou \textit{Mega}.

\subsubsection{Les différents pins de branchement}

\illustrate{arduino_uno_r3.png}{Branchements sur une carte}

\begin{itemize}
	\item En \textbf{noir}, les pins de masse.
		  Ce sont les bornes négatives du circuit.
		  Les trois sont les mêmes et indistinctes.
	\item En \textbf{rouge}, les pins d'alimentation.
		  Du courant provient de ces pins, ce sont les bornes positives.
		  Il y a deux tensions disponibles mais vous n'allez utiliser que celle en 5V.
	\item En \textbf{bleu}, les pins d'entrée/sortie numérique.
		  Ces pins permettent de lire et écrire de l'information binaire (HIGH ou LOW).
		  Il est possible d'écrire en analogique sur les pins avec le symbole \boldmath$\sim$.
		  Attention, le pin 0 n'est pas utilisable en lecture.
	\item En \textbf{vert}, les pins d'entrée analogique.
		  Ces pins permettent de lire de l'information analogique.
		  L'écriture n'est pas possible.
	\item Les autres pins ne seront pas utiles pour cet atelier.
\end{itemize}

\subsubsection{Le langage Arduino}

L'intérêt de ces cartes est de pouvoir les programmer afin de contrôler les éléments qui y sont branchés.
Cela se fait avec le langage Arduino à la syntaxe très proche de C et C++.
Vous pouvez retrouver la référence \doclink{https://www.arduino.cc/reference/en/}{ici}, n'hésitez pas à aller vous renseigner fréquemment en cas de besoin.
Diverses bibliothèques sont également disponibles en fonction des composants utilisés.

\br
Au minimum, vous aurez besoin d'utiliser :
\begin{itemize}
	\item \textbf{pinMode(pin, mode)} pour définir le mode de fonctionnement d'un pin (INPUT ou OUTPUT)
	\item \textbf{digitalRead(pin)} pour lire la valeur numérique d'un pin (HIGH ou LOW)
	\item \textbf{digitalWrite(pin, value)} pour écrire une valeur numérique (HIGH ou LOW) sur un pin
	\item \textbf{analogRead(pin)} pour lire la valeur analogique d'un pin
	\item \textbf{analogWrite(pin, value)} pour écrire une valeur analogique sur un pin	
\end{itemize}

\br
Deux fonctions sont présentes de base lorsque vous placez une carte sur Tinkercad :
\begin{itemize}
	\item \textbf{setup()} qui est appelée lorsque la carte est mise sous tension
	\item \textbf{loop()} qui est appelée en continu
\end{itemize}

\hint{Il est fortement recommandé d'ajouter un délai de quelques millisecondes avant la fin à l'aide de la fonction \textbf{delay(temps)}}


\clearpage
\subsection{Application}
Maintenant que vous avez des ressources à disposition, c'est le moment de se lancer !
Voici quelques exercices pour résumer ce qui a été présenté.
Il est recommandé de les faire à la suite pour construire un système de plus en plus complexe.
Les solutions sont en annexe si nécessaire.

\subsubsection{Envie de buzzer...}
Matériel : \textit{une carte arduino, un bouton poussoir, une résistance, un buzzer piezo}
\\
\textbf{Lorsque le bouton est pressé, il faut que le piezo émette un son.}

\subsubsection{Que la lumière soit !}
Matériel : \textit{le résultat du précédent, deux LEDs (une verte et une rouge), deux résistances}
\\
\textbf{Lorsque le bouton est relâché, la LED rouge doit être allumée}
\\
\textbf{Lorsque le bouton est pressé, la LED verte doit être allumée}

\subsubsection{Bientôt DJ ?}
Matériel : \textit{le résultat du précédent, un potentiomètre}
\\
\textbf{Le potentiomètre doit contrôler la fréquence du buzzer piezo}


\subsection{Mini-projet : arduiPONG}
Vous arrivez à la fin de cet atelier avec désormais de bonnes connaissances pour vous aventurer par vous-mêmes !
Voici une proposition de petit projet que vous devriez être en mesure de faire : recréer votre version du jeu Pong en utilisant des entrées/sorties physiques.
L'intérêt ici est dans la partie électronique, vous pourrez rédiger le code de votre côté après si vous le souhaitez.
\\
Voici la description :
\begin{itemize}
	\item Un bouton principal permet de lancer le jeu
	\item Deux joueurs s'affrontent et commandent leur raquette avec deux boutons chacun (haut et bas)
	\item Chaque joueur a une LED qui s'allume lorsqu'un point est gagné
	\item L'affichage se fait à l'aide d'un écran LCD, suivez la \doclink{https://docs.arduino.cc/learn/electronics/lcd-displays}{documentation} pour le brancher et l'utiliser
	\item Vous pouvez rajouter un buzzer piezo qui émettra des sons quand un but est marqué, au décompte du départ, à chaque rebond, etc...
\end{itemize}

Avec ce système, vous devriez utiliser les 14 pins numériques avec :
\begin{itemize}
	\item 5 boutons
	\item 2 LEDs
	\item 1 écran LCD
	\item 1 buzzer piezo
	\item 1 potentiomètre (si vous voulez pouvoir régler le contraste de l'écran)
	\item 3 résistances 220$\Omega$
	\item 5 résistances 10k$\Omega$
\end{itemize}

\br
Une proposition de circuit est disponible en annexe.
Le code n'est pas inclus car il relève davantage d'un problème de programmation et se trouve hors du spectre de cet atelier.
Libre à vous d'implémenter les fonctionnalités que vous désirez !